\newgeometry{left=3cm,top=2cm}
\chapter*{Abstract}

Questa tesi si concentra sul problema del coordinamento di più UAV per il rilevamento e il tracking di target distribuiti, in diversi contesti tecnologici e ambientali. 
\'E stato implementato un ambiente di simulazione, basato sulla tecnologia UAV disponibile in commercio e su scenari reali. 
L'approccio analizzato si basa sul concetto di comportamento dello sciame in sistemi multi-agente, cioè un team di UAV autoformato e auto coordinato che si adatta a layout ambientali specifici della missione. 
La formazione e la coordinazione degli sciami sono ispirati da meccanismi biologici.
Questi meccanismi, opportunamente combinati, permettono di trovare il giusto equilibrio tra ricerca globale (\textit{exploration}) e ricerca locale (\textit{exploitation}) nell'ambiente. 
Il simulatore è provvisto di un modulo software per l'integrazione di algoritmi basati sulle strategie presenti in letteratura.
L'obiettivo del simulation tested proposto è quello di condurre una valutazione delle performance di tali strategie e di effettuare un'analisi comparativa, basata su diversi test di simulazione, per identificare i punti chiave da sfruttare e migliorare nei lavori futuri. 
Le strategie analizzate in questa tesi sono: (i) Algoritmo Sciadro basato su meccanismi biologici come il flocking e la stigmergia; (ii) Firefly-based team strategy for robot recruitment, basato sul comportamento delle lucciole; (iii) Particle swarm optimization for robot recruitment, basato sul comportamento di agenti in stormo; (iv) Artificial bee colony algorithm for robot recruitment, basato invece sul comportamento delle api nella ricerca di cibo.
L'adattamento dello sciame può essere sostenuto da algoritmi evolutivi con l'obiettivo di massimizzare il numero di bersagli tracciati durante una missione o di ridurre al minimo il tempo per la scoperta del bersaglio. 
[ --- BOZZA --- I risultati sperimentali dimostrano che l'ambiente permette una valutazione qualitativa e quantitativa delle performance ed un'importante ottimizzazione parametrica delle strategie bio-ispirate analizzate --- BOZZA --- ]

\restoregeometry