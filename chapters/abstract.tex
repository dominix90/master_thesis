\newgeometry{left=3cm,top=2cm}
\chapter*{Abstract}

Questa tesi si concentra sul problema del coordinamento di più UAV per il rilevamento e il tracking di target distribuiti, in diversi contesti tecnologici e ambientali. 
È stato sviluppato e rilasciato pubblicamente un ambiente di simulazione, basato sulla tecnologia UAV disponibile in commercio e su scenari reali. 
L'approccio proposto si basa sul concetto di comportamento dello sciame in sistemi multi-agente, cioè un team di UAV autoformato e auto coordinato che si adatta a layout ambientali specifici della missione. 
La formazione e la coordinazione degli sciami sono ispirati da meccanismi biologici come il floccaggio e la stigmergia. 
Questi meccanismi, opportunamente combinati, permettono di trovare il giusto equilibrio tra ricerca globale (\textit{exploration}) e ricerca locale (\textit{exploitation}) nell'ambiente. 
L'adattamento dello sciame può essere basato su algoritmi evolutivi con l'obiettivo di massimizzare il numero di bersagli tracciati durante una missione o di ridurre al minimo il tempo per la scoperta del bersaglio. 
I risultati sperimentali dimostrano che l'ambiente permette di estendere e superare gli approcci esistenti in letteratura. 
È stata effettuata un'analisi comparativa, basata su diversi test di simulazione sul banco di prova sviluppato, per identificare i punti chiave da sfruttare e migliorare nei lavori futuri.

\restoregeometry